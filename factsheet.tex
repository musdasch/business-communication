%% Erläuterungen zu den Befehlen erfolgen unter
%% diesem Beispiel.

\documentclass{scrartcl}

\usepackage[utf8]{inputenc}
\usepackage[T1]{fontenc}
\usepackage{lmodern}
\usepackage[ngerman]{babel}

\title{Fakten Blatt}
\author{Pratric Furler, Tim Vögtli}
\date{2. September 2017}
\begin{document}

\maketitle

\section{Name des Programms/Frameworks}
LaTex

\section{Beschreibung}
LaTex ist eine Software, der die Programmiersprache Tex verwendet.

\section{Ursprung (Buchdruck, Desktop Publishing ect.)}

Das Basis-Programm von LaTex ist Tex und wurde von Donald E. Knuth ab 1977 entwickelt.
Das Wort LaTex setzt sich aus Lampert (Leslie Lampert) und Tex zusammen.
LaTex ist eine Sammlung von nützliche Tex-Makros für den durchschnittlichen Anwender.
Die neueren Versionen von LaTex ab 1990 wurde mittlerweile von verschiedene Programmiere mitentwicklet.

\section{Voraussetzungen (Betriebssystem, Browser etc.)}

LaTex ist auf Linux, Mac OS, Windows und Online verwendbar.

\section{Installation (mit Linux Distribution, Installer etc.)}

Auf latex-projekt.org sind z.B. alle Versionen erhältlich.
\begin{itemize}
	\item Linux - TeX Live
	\item Mac OS - MacTex oder BasicTex
	\item Windows - MiKTex, proTeXt oder TeX Live
	\item Online - Papeeria, Overleaf, ShareLaTex oder Datazar
\end{itemize}

\section{Layoutvorlagen}
Es gibt unzählige Vorlagen im Internet (z.B. auf ShareLaTeX.com/templates).
\begin{itemize}
	\item bibtex
	\item Easy Book
	\item Exam
	\item Business Card
|\end{itemize}

\section{Unterstützte Strukturelemente}
Unterstützt wird:
\begin{itemize}
	\item Aufbau der Quelldatei
	\item Aufbau der Gliederungsbefehle
	\item Aufbau des Verzeichnisses
	\item Verweise innerhalb des Dokumentes
|\end{itemize}

\section{Verlinkung von externen Quellen}
Innerhalb vom Dokument sind unterschiedliche Verlinkungsmöglichekeiten möglich.
Darunter auch Links ins Internet, im Dokument selber und auf exkternen Dokumenten.

\section{Importformate (eps, png, jpg etc.)}
\begin{itemize}
	\item eps - Bilder werden unterstützt mit einem zusätzlichem Packet (epstopdf)
	\item png - empfohlen für Diagramme
	\item jpg - empfohlen für Bilder
	\item pdf - kann auch Bilder enthalten
|\end{itemize}

\section{Ausgabeformate(pdf, doc, docx, ps etc.)}
LaTex gibt die Dokumentation im der Form von pdf, HTML oder PostScript aus.
Mit unterschiedlichen Konverter ist es jedoch möglich, Word-Textdateien, Exel-Tabellen und weiter Formaten auszugeben.

\section{Druckformate (DIN A-Serie, Flyer, Prospekte, Buch, Manuel etc.)}
Der Druckformat ist im DIN, US-amerikanische typografischen Konventionen oder KOMA-Script möglich.
Es gibt aber zusätzliche Pakete und Klassen die weitere Formaten erlauben.

\section{Mehrbenutzerfähigkeit (Aufteilung der Dokumente, Konfliktbewältigung beim Zusammenfügen,
Versionierungsmöglichkeiten etc.)}
LaTex kann z.B. über Git verwendet werden, um gleichzeitige Anpassungen vom Dokument zu unterstützen.

\section{Kosten (Einzelplatz, Team à 5 Personen)}
LaTex ist Open Source, jedoch gibt es Tools, welche LaTex unterstützen und auf Spenden angewiesen sind.

\section{Quellen}

\end{document}