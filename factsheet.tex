%% Erläuterungen zu den Befehlen erfolgen unter
%% diesem Beispiel.

\documentclass{scrartcl}

\usepackage[utf8]{inputenc}
\usepackage[T1]{fontenc}
\usepackage{lmodern}
\usepackage[ngerman]{babel}
\usepackage{amsmath}

\title{Fakten Blatt}
\author{Pratric Furler, Tim Vögtli}
\date{2. September 2017}
\begin{document}

\maketitle

\section{Name des Programms/Frameworks}
LaTex
\section{Beschreibung}
LaTex ist eine Software, der die Programmiersprache Tex verwendet.
\section{Ursprung (Buchdruck, Desktop Publishing ect.)}
Das Wort LaTex setzt sich aus Leslie Lampert (La) und Tex zusammen.

\section{Voraussetzungen (Betriebssystem, Browser etc.)}

\section{Installation (mit Linux Distribution, Installer etc.)}

\section{Layoutvorlagen}

\section{Unterstützte Strukturelemente}

\section{Verlinkung von externen Quellen}

\section{Importformate (eps, png, jpg etc.)}

\section{Ausgabeformate(pdf, doc, docx, ps etc.)}

\section{Druckformate (DIN A-Serie, Flyer, Prospekte, Buch, Manuel etc.)}

\section{Mehrbenutzerfähigkeit (Aufteilung der Dokumente, Konfliktbewältigung beim Zusammenfügen,
Versionierungsmöglichkeiten etc.)}

\section{Kosten (Einzelplatz, Team à 5 Personen)}

\section{Quellen}

\end{document}