%% The document class.
\documentclass{article}

%% ---- Packages ----
%% --- Font ---
\usepackage[utf8]{inputenc}						%% Sets the character set to utf-8.
\usepackage[T1]{fontenc}						%% Expands the character addresses.
\usepackage[scaled]{helvet}						%% Sets the font family to helvet.
\usepackage{textcomp}							%% It pro­vides many text sym­bols.

%% --- Language ---
\usepackage[ngerman]{babel} 					%% Sets the language to german. It is used for the date.

%% --- Last Page ---
\usepackage{lastpage} 							%% Expands the comands with \pageref{LastPage}.

%% --- Layout ---
\usepackage{geometry} 							%% It is used to set the page layout.
\usepackage{fancyhdr} 							%% It is used to set header and footer.
\usepackage{graphicx}							%% Enables to use Images.

%% --- Debug ---
%%\usepackage{showframe}						%% It is used to show the segments.

%% ---- Meta data ----
\title{Fakten Blatt - \LaTeX}					%% Sets the title of the document.
\author{Patrick Furler, Tim Vögtli}				%% Sets the authors of the document.
\date{\today}									%% Sets the date.

%% Copy the title and the autor for using in the header and footer.
\makeatletter
	\let\runauthor\@author
	\let\runtitle\@title
\makeatother


%% ---- Style ----
\geometry{
	a4paper,									%% Sets the format to A4.
	tmargin=25mm,								%% Sets a margin on top.
	bmargin=25mm,								%% Sets a margin on the bottom
	lmargin=30mm,								%% Sets a margin left from the content.
	rmargin=20mm,								%% Sets a margin right from the content.
	headheight=34.78244pt
}

\renewcommand{\headrulewidth}{0pt}				%% Hides the line on top. 
\renewcommand{\footrulewidth}{0pt}				%% Hides the line on bottom.

\renewcommand\familydefault{\sfdefault}			%% Sets the default font family to a sans serif family.

\graphicspath{{./img/}} 						%% Path to the images.

\pagestyle{fancy}								%% Sets the page style to fancy.
\fancyhf{}										%% Clear header and footer.
\lhead{\includegraphics[width=2cm]{logo}}		%% Add a logo to the left header.
\chead{\fontsize{8}{10} \selectfont \runtitle}	%% Add the title to the centered header.

\lfoot{\fontsize{8}{10} \selectfont \runauthor}	%% Add the authors to the left footer.
\cfoot{\fontsize{8}{10} \selectfont \today}		%% Add the date to the centered footer.
\rfoot{\fontsize{8}{10} \selectfont Seite \thepage\ von \pageref{LastPage}}	%% Add the page nummber to the right footer.


\begin{document}								%% begin document

\maketitle 										%% Makes the title
\thispagestyle{fancy}							%% Shows the header and the footer on The front page.

\section{Name des Softwarepaket}
\LaTeX

\section{Beschreibung}
\LaTeX\ ist eine Software, der die Programmiersprache \TeX\ verwendet. \LaTeX\ funktioniert nicht nach dem \textbf{WYSIWYG} Prinzip sondern ist wie HTML eine Auszeichnungssprache.

\section{Ursprung}

Das Basis-Programm von LaTex ist \TeX\ und wurde von Donald E. Knuth ab 1977 entwickelt. Das Wort \LaTeX\ setzt sich aus \textbf{La}mpert (Leslie Lampert) und \TeX\ zusammen. \LaTeX\ ist eine Sammlung von nützliche \TeX-Makros für den durchschnittlichen Anwender. Die neueren Versionen von LaTex ab 1990 wurde mittlerweile von verschiedene Programmiere mitentwickelt.

\section{Voraussetzungen}

\LaTeX\ kann als native Installation auf allen gängigen Plattformen eingesetzt werden oder via einer Online Plattform wie overleaf.com verwendet werden.

\section{Installation}

Auf latex-projekt.org sind z.B. alle Versionen erhältlich.
\begin{itemize}
	\item Linux - TeX Live
	\item Mac OS - MacTex oder BasicTex
	\item Windows - MiKTex, proTeXt oder TeX Live
	\item Online - Papeeria, Overleaf, ShareLaTex oder Datazar
\end{itemize}

\section{Layoutvorlagen}
Es gibt unzählige Vorlagen im Internet (z.B. auf ShareLaTeX.com/templates).
\begin{itemize}
	\item bibtex
	\item Easy Book
	\item Exam
	\item Business Card
\end{itemize}

\section{Unterstützte Strukturelemente}
Unterstützt wird:
\begin{itemize}
	\item Aufbau der Quelldatei
	\item Aufbau der Gliederungsbefehle
	\item Aufbau des Verzeichnisses
	\item Verweise innerhalb des Dokumentes
\end{itemize}

\section{Verlinkung}
Innerhalb vom Dokument sind unterschiedliche Verankerungsmöglichkeiten möglich. Darunter auch Links ins Internet, im Dokument selber und auf externen Dokumenten.

\section{Importformate}
\begin{itemize}
	\item eps - Bilder werden unterstützt mit einem zusätzlichem Paket (epstopdf)
	\item png - empfohlen für Diagramme
	\item jpg - empfohlen für Bilder
	\item pdf - kann auch Bilder enthalten
\end{itemize}

\section{Ausgabeformate}
\LaTeX\ gibt die Dokumentation im der Form von PDF, HTML oder PostScript aus. Mit unterschiedlichen Konverter ist es jedoch möglich, Word-Textdateien, Exel-Tabellen und weiter Formaten auszugeben.

\section{Druckformate}
Das Druckformat ist im DIN, US-amerikanische typografischen Konventionen oder KOMA-Script möglich. Es gibt aber zusätzliche Pakete und Klassen die weitere Formate erlauben.

\section{Mehrbenutzerfähigkeit}
\LaTeX\ kann z.B. mit einem GIT System oder einer Online Plattform verwendet werden, um das gleichzeitige Arbeiten an einem Dokument zu unterstützen. 

\section{Kosten}
\LaTeX\ ist Open Source, jedoch gibt es Tools, welche \LaTeX\ unterstützen und auf Spenden angewiesen sind.

\section{Quellen}
\begin{itemize}
	\item https//latex-projekt.org
	\item https//de.sharelatex.com
	\item https//de.wikipedia.org/wiki/LaTex
\end{itemize}



\end{document}								%% end document