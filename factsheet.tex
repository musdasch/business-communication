%% Erläuterungen zu den Befehlen erfolgen unter
%% diesem Beispiel.
\documentclass{article}

%% Packages
\usepackage[utf8]{inputenc}
\usepackage[scaled]{helvet}
\usepackage[T1]{fontenc}
\usepackage{lmodern}
\usepackage[ngerman]{babel}
\usepackage{textcomp}

\usepackage{lastpage}

\usepackage{geometry}
\usepackage{fancyhdr}
\usepackage{graphicx}
%%\usepackage{showframe}

%% ---- Meta data ----
\title{Fakten Blatt - \LaTeX{}}

\author{Pratric Furler, Tim Vögtli}
\date{\today}

\makeatletter
	\let\runauthor\@author
	\let\runtitle\@title
\makeatother


%% ---- Style ----
\geometry{
	a4paper,
	tmargin=25mm,
	bmargin=25mm,
	lmargin=30mm,
	rmargin=20mm,
	headheight=34.78244pt
}

\renewcommand{\headrulewidth}{0pt}
\renewcommand{\footrulewidth}{0pt}

\renewcommand\familydefault{\sfdefault}

\graphicspath{{./img/}} %% the graphicspath has to be changed.

\pagestyle{fancy}
\fancyhf{}
\lhead{\includegraphics[width=2cm]{logo}}
\chead{\fontsize{8}{10} \selectfont \runtitle}

\lfoot{\fontsize{8}{10} \selectfont \runauthor}
\cfoot{\fontsize{8}{10} \selectfont \today}
\rfoot{\fontsize{8}{10} \selectfont Seite \thepage\ von \pageref{LastPage}}


\begin{document}

\maketitle
\thispagestyle{fancy}

\section{Name des Softwarepaket}
\LaTeX

\section{Beschreibung}
\LaTeX\ ist eine Software, der die Programmiersprache \TeX\ verwendet. \LaTeX\ funktioniert nicht nach dem \textbf{WYSIWYG} Prinzip sondern ist wie HTML eine Auszeichnungssprache.

\section{Ursprung}

Das Basis-Programm von LaTex ist \TeX\ und wurde von Donald E. Knuth ab 1977 entwickelt. Das Wort \LaTeX\ setzt sich aus \textbf{La}mpert (Leslie Lampert) und \TeX\ zusammen. \LaTeX\ ist eine Sammlung von nützliche \TeX-Makros für den durchschnittlichen Anwender. Die neueren Versionen von LaTex ab 1990 wurde mittlerweile von verschiedene Programmiere mitentwickelt.

\section{Voraussetzungen}

\LaTeX\ kann als native Installation auf allen gängigen Plattformen eingesetzt werden oder via einer Online Plattform wie overleaf.com verwendet werden.

\section{Installation}

Auf latex-projekt.org sind z.B. alle Versionen erhältlich.
\begin{itemize}
	\item Linux - TeX Live
	\item Mac OS - MacTex oder BasicTex
	\item Windows - MiKTex, proTeXt oder TeX Live
	\item Online - Papeeria, Overleaf, ShareLaTex oder Datazar
\end{itemize}

\section{Layoutvorlagen}
Es gibt unzählige Vorlagen im Internet (z.B. auf ShareLaTeX.com/templates).
\begin{itemize}
	\item bibtex
	\item Easy Book
	\item Exam
	\item Business Card
|\end{itemize}

\section{Unterstützte Strukturelemente}
Unterstützt wird:
\begin{itemize}
	\item Aufbau der Quelldatei
	\item Aufbau der Gliederungsbefehle
	\item Aufbau des Verzeichnisses
	\item Verweise innerhalb des Dokumentes
|\end{itemize}

\section{Verlinkung}
Innerhalb vom Dokument sind unterschiedliche Verankerungsmöglichkeiten möglich. Darunter auch Links ins Internet, im Dokument selber und auf externen Dokumenten.

\section{Importformate}
\begin{itemize}
	\item eps - Bilder werden unterstützt mit einem zusätzlichem Paket (epstopdf)
	\item png - empfohlen für Diagramme
	\item jpg - empfohlen für Bilder
	\item pdf - kann auch Bilder enthalten
|\end{itemize}

\section{Ausgabeformate}
\LaTeX\ gibt die Dokumentation im der Form von PDF, HTML oder PostScript aus. Mit unterschiedlichen Konverter ist es jedoch möglich, Word-Textdateien, Exel-Tabellen und weiter Formaten auszugeben.

\section{Druckformate}
Das Druckformat ist im DIN, US-amerikanische typografischen Konventionen oder KOMA-Script möglich. Es gibt aber zusätzliche Pakete und Klassen die weitere Formate erlauben.

\section{Mehrbenutzerfähigkeit}
\LaTeX\ kann z.B. mit einem GIT System oder einer Online Plattform verwendet werden, um das gleichzeitige Arbeiten an einem Dokument zu unterstützen. 

\section{Kosten}
\LaTeX\ ist Open Source, jedoch gibt es Tools, welche \LaTeX unterstützen und auf Spenden angewiesen sind.

\section{Quellen}
\begin{itemize}
	\item https//latex-projekt.org
	\item https//de.sharelatex.com
	\item https//de.wikipedia.org/wiki/LaTex
|\end{itemize}



\end{document}